\documentclass[10pt]{article}
\bibliographystyle{plos2009}
\makeatletter
\renewcommand{\@biblabel}[1]{\quad#1.}
\makeatother

\date{}

\pagestyle{myheadings}

\newcommand{\fixme}[2]{\textsc{\textbf{{#1}: {#2}}}}
\newcommand{\recommend}[1]{\textit{#1}}
\newcommand{\withurl}[2]{{#1}\footnote{\texttt{#2}}}

\begin{document}

\begin{flushleft}
{\Large
\textbf{Ten Simple Rules for Making Research Software More Robust}
}

{Morgan~Taschuk}$^{1,\ast}$,
{Greg~Wilson}$^{2}$
\\
1) Ontario Institute for Cancer Research / morgan.taschuk@oicr.on.ca
\\
2) Software Carpentry Foundation / gvwilson@software-carpentry.org
\\
$\ast$ Corresponding author.
\end{flushleft}


\begin{quote}
\begin{center}\textbf{Abstract}\end{center}
\fixme{GW}{Write abstract}
\end{quote}

\section*{Introduction}

As a relatively young field, bioinformatics is full of newly developed software.
\fixme{MT}{examples here}
Efforts such as
\withurl{the ELIXIR tools and data registry}{http://dx.doi.org/10.1093/nar/gkv1116}
and the
\withurl{Bioinformatics Links Directory}{https://bioinformatics.ca/links\_directory/}
\cite{brazas2012}
have made efforts towards cataloguing it: as of May 2016, the former
has 2500 entries and the latter 1700, and those numbers are constantly
growing as both trainees and experienced practitioners produce new
software to support their research.

Typically, this software is initially developed by one person, and may
produce excellent results in their hands. But what happens after that
student leaves that lab and someone else wants to use it? Everyone with
a few years of experience feels a tremor of fear when told, ``Use
{\textless}graduated student{\textgreater}'s code to analyze your
data''. Often, that software will be undocumented and work in unexpected
ways (if it works at all without substantial modification). Equally
often, the potential new user winds up shaking their fist and cursing
the author's name. She then has two choices: hack the existing code to
make it work for her, or start over.

The root cause of this problem is that most of the software researchers
produce isn't \emph{robust}. The difference between running and being
robust is the difference between ``works for me on my machine'' and
``works for other people on a cluster I've never used''. In particular,
robust software:

\begin{itemize}
\item
  Is kept under version control.
\item
  Can be installed on systems outside the original institution
\item
  Works for users other than the original author
\item
  Has well-defined input and output formats
\item
  Has documentation that describes what its dependencies are, how to
  install it, and what the options are.
\item
  Comes with enough tests to show that it actually runs.
\end{itemize}

These are all necessary steps toward creating a reusable library that
can be shared through a site like CPAN or CRAN, and apply to both
closed-source and open-source software. They do not depend on specific
languages, libraries, packages, documentation styles, or operating
systems. Whether the aim is as simple as sharing the code with
collaborators or as complex as using the software in a production
analysis environment, increasing the robustness of your software
decreases headaches all around.

Note: we do not recommend that these rules be applied to \emph{every}
coding effort. The vast majority of code produced in the marathon of a
graduate thesis is ``throw-away'' code that is used once to answer a
specific question related to a specific dataset. However, once that
little script is dragged out three or four times for slightly different
purposes, it may be time to apply ten simple rules for robust software.
As the saying goes, not everything worth doing is worth doing right
away.

\section{Have a README that explains in a few lines what the software does and what its dependencies are.}

The README is the first stop for any potential users interested in your
software. At a minimum, it needs to provide or point to everything a new
user needs to get started, where they can turn to for help, and which
licenses apply to the software package. Exhaustive details regarding
parameters and usage are not usually necessary in a README if they are
present in usage (\#2), although a working example using test data (per
\#9) is always appreciated.

\textbf{Explain what the software does:} At the beginning of the README,
explain what the software does in one or two sentences. The description
does not need to be long or detailed. There's nothing more frustrating
than spending the time to download and install some software only to
find out that it doesn't do what you thought it did.

\begin{verbatim}
Debarcer (De-Barcoding and Error Correction) is a package for
working with next-gen sequencing data that contains molecular
barcodes.
\end{verbatim}

\textbf{List required dependencies:} Often, software depends on very
specific versions of libraries, modules, or operating systems. This is
entirely reasonable as long as it is properly documented. Often,
multiple libraries exist with the same or very similar names, so
either provide the commands necessary to download the dependencies or
link to the software homepage. Include the version number for each
dependency.  Especially if you use older versions, include links where
the packages can be downloaded. Package managers like apt, pip and
homebrew stop offering older packages after a few years.
\fixme{MT}{Is this true of pip and Homebrew?}

If your dependency is to an internal package that is not available on
the internet, you have several options depending on the sensitivity of
the code in question. If it is plain text, you can add it directly to
you repository with appropriate attribution. If the dependency is a
binary, we recommend using a
\withurl{binary repository manager}{https://en.wikipedia.org/wiki/Binary\_repository\_manager}
such as Artifactory or ProGet. These managers keep versioned copies of
software at constant URLs so they can be downloaded as long as the
manager continues to run. As a last resort, you can place it at an
internal location on shared disk, remove all write permissions, and
link to it from your README, although this method is heavily
discouraged because of the potential for the directory to go missing
because of factors outside the developer's control.

\textbf{Installation instructions:} If the software needs to be compiled
or installed, list those instructions in the readme. New users may not
be familiar with your build system, even if it is \texttt{make}. Also
mention here if you recommend they use a pre-compiled binary instead
through a system such as pip or apt.

\textbf{Input and output files}: All possible input and output files
should be listed in this section. Do the files conform to a particular
industry standard, an extension of an existing format, or is it your own
format? If using a standard format, link to the specification and
version. If you extend the standard or have your own format, define it
here explicitly, listing all the required fields and acceptable values.
(You get bonus points if you include a script to convert between
standard format and your file format). If there is no rigorous format
(such as with log files), show an example file, or the first few lines,
and explain what the sections mean.

Input files and their formats are included in most documentation.
However, the definitions of the output files are often missing. In
addition to the expected output, software will often produce
intermediate files, auxilliary files, and log files. We believe
\emph{all} output files should be listed in the README. Log and
auxilliary files are often full of valuable information that can be
mined for the user's specific purpose. Even if the files are considered
self-explanatory. Sometimes your users will need to answer a question of
the format, ``Does X tell you the percentage of reads trimmed to remove
adapter sequences?'' and you can check the documentation and confidently
say ``yes, it is in the log file''.

\textbf{Attributions and licensing:} Attributions are how you credit
your main contributors; licenses are how you want others to use and
credit your software. Both are important in your README. Leave no
question in anyone's mind about whether your software can be used
commercially, how much modification is permitted, and how other software
needs to attribute to you. If your software is not open source, include
a statement here. Attributions can also contain a list of `expert' users
that can be contacted if new users have problems with the software. (for
better or for worse)

\begin{itemize}
\item
  This is readable \emph{before} the software is installed (or even
  downloaded).
\item
  Should also include (or better yet, point to) the license for the
  software, so that (potential) users will know what they're allowed to
  do.
\item
  \fixme{GW}{example from real software:\\ https://github.com/dib-lab/khmer/blob/master/README.rst}
\end{itemize}

\section{Print usage information when launching from the command line that explains the software's features.}

Users who run your software after installation may not have access your
well-crafted README (or may not have bothered to read it). Usage
information provides their first line of help.

Ideally, usage is a terse, informative command-line help message that
guides the user in the correct use of your software. Terseness is
important: usage that extends for multiple screens, especially when
printed to standard error instead of standard output (where it can
easily be paged), is a nuisance, and is as unlikely to be read as the
README file.

Usage should provide all of the information necessary to run the
software. It is usually invoked either by running the software without
any arguments; running the software with incorrect arguments; or by
explicitly choosing a help or usage option.

An example of good usage is the Unix \texttt{mkdir} command, which makes
new directories:

\begin{small}
\begin{verbatim}
$ mkdir --help
Usage: mkdir [OPTION]... DIRECTORY...
Create the DIRECTORY(ies), if they do not already exist.

Mandatory arguments to long options are mandatory for short options too.
  -m, --mode=MODE   set file mode (as in chmod), not a=rwx - umask
  -p, --parents     no error if existing, make parent directories as needed
  -v, --verbose     print a message for each created directory
  -Z, --context=CTX  set the SELinux security context of each created
                      directory to CTX
      --help     display this help and exit
      --version  output version information and exit

Report mkdir bugs to bug-coreutils@gnu.org
GNU coreutils home page: <http://www.gnu.org/software/coreutils/>
General help using GNU software: <http://www.gnu.org/gethelp/>
For complete documentation, run: info coreutils 'mkdir invocation'
\end{verbatim}
\end{small}

There is no standard format for usage statements, but good ones share
several features:

\textbf{The syntax for running the program}: This defines the relative
location of optional and required flags and arguments for execution, and
includes the name of the program. Arguments in {[}square brackets{]}
tend to be optional. Multiple periods (e.g. ``{[}OPTION{]}\ldots{}'')
indicate that more than one can be provided.

\textbf{A text description of its purpose}: Similar to the README, the
description reminds users of the software's primary function.

\textbf{Most commonly used flags, a description of each flag, and the
default value}: Not all flags need to appear in the usage, but the most
commonly used ones should be listed here. Users will rely on this for
quick reference when working with your software.

\textbf{Where to find more information}: Whether an email address, web
site or manual page, there should be an indication where the user can go
to find more information about the software.

\textbf{Printed to standard output} : So that it can be piped into
\texttt{less}, searched with \texttt{grep}, or compared to the previous
version's help with \texttt{diff}.

\textbf{Exit with an appropriate exit code}: When usage is invoked by
providing incorrect parameters, the program should exit with a non-zero
code to indicate an error. However, when help is explicitly requested,
the software should not exit with an error, because requesting help is
sometimes used to verify that a dependency is available.

\section{Do not require root or other special privileges.}

Root (also known as ``superuser'' or ``admin'') is a special account on
a computer that has (among other things) the power to modify or delete
system files and user accounts. Conversely, files and directories owned
by root usually \emph{cannot} be modifed by normal users.

Installing or running a program with root privileges is often
convenient, since doing so automatically bypasses all those pesky safety
checks that might otherwise get in the user's way. However, those checks
are there for a reason: scientific software packages may not
intentionally be malware, but one small bug or over-eager file-matching
expression can certainly make them behave as if they were. Outside of
very unusual circumstances, packages should therefore not require root
privileges to set up or use.

Another reason for this rule is that users may want to try out a new
package before installing it system-wide on a cluster. Requiring root
privileges will frustrate such efforts, and thereby reduce uptake of the
package. Requiring that software be installed under its own user account
(e.g., that \texttt{packagename} be made a user, and all of the
package's software be installed in that ``user's'' space) is similarly
limiting, and makes side-by-side installation of multiple versions of
the package more difficult.

Developers should therefore allow packages to be installed in an
arbitrary location, e.g., under a user's home directory in
\texttt{\textasciitilde{}/packagename}, or in directories with standard
names like \texttt{bin}, \texttt{lib}, and \texttt{man} under a chosen
directory. If the first option is chosen, the user may need to modify
her search path to include the package's executables and libraries, but
this can (more or less) be automated, and is much less risky than
setting things up as root.

\section{Allow configuration of all useful parameters from the command line.}

You know what they say about assumptions*. As soon as you make one
assumption about how a software should work, a use case will come along
that will require you to change the assumption.

Every useful parameter should be configurable on the command line.
Useful parameters are those that a user will need to modify to suit
their computer, dataset or application. Providing parameters on the
command line increases the flexibility and usability of the program. You
may have determined early on that 0.58 is an optimal seed for your
original dataset, but that doesn't mean that is the best seed for every
case. Being able to change parameters on the fly to determine if and how
they change the results is important as your software gains more users,
facilitating exploratory analysis and parameter sweeping. Keep in mind
that if a parameter can be adjusted, users will want to be able to turn
off the feature entirely to have a baseline comparison.

The list of useful parameters is software-specific and so cannot be
provided here, but here is a short list of common useful parameters.

\begin{itemize}
\item
  Input and reference files/directories
\item
  Output files/directories
\item
  Filtering
\item
  Tuning (e.g.~alphas and gammas)
\item
  Seeds
\item
  Any alternatives that you've built-in, e.g.~compress results, use a
  different calculation, verbose output, etc.
\end{itemize}

When the software starts, it should echo all parameters and software
versions to standard out or a log file alongside the results. This
feature supports greater reproducibility because any result can be
replicated with only the previous output files as reference.

You can set reasonable default values to reduce the length of the
execution command as long as any parameters given on the command line
override those values. Configuration files should be used in preference
to hardcoding the defaults directly. Only values that are unlikely to
change between runs belong in the config file, such as dependencies,
servers, version numbers, network drives, and any other defaults for
your lab or institutions. Configuration files can be in a standard
location, e.g. \texttt{.packagerc} in the user's home directory or
provided on the command line as an additional argument.

We caution against overusing configuration files. When a user needs to
locate, open, change and save a file in order to change a parameter, the
import of the change seems larger and discourages experimentation. Since
all parameters should be echoed in the results, config files must be
cleaned up after execution completes or they occupy valuable disk space.
Use a default configuration file and specify all other parameters on the
command line.

* They make fools out of you and me.

\section{Eliminate hard-coded paths.}

It's easy to write software that reads input from a file called
\texttt{mydata.csv}, but also very limiting. If a colleague asks you to
process her data, you must either overwrite your data file (which is
risky) or edit your code to read \texttt{otherdata.csv} (which is also
risky, because there's every likelihood you'll forget to change the
filename back, or will change three uses of the filename but not a
fourth).

Hard-coding filenames in a program also makes the software harder to run
in other environments. If your package is installed on a cluster, for
example, the user's data will almost certainly \emph{not} be in the same
directory as the software, and the folder
\texttt{C:\textbackslash{}users\textbackslash{}yourname\textbackslash{}}
will probably not even exist.

For these reasons, users should be able to set the names and locations
of input and output files as command-line parameters. This rule applies
to reference data sets as well as the user's own data: if a user wants
to try a new gene identification algorithm using a different set of
genes as a training set, she should not have to edit the software to do
so.

A corollary to this rule is that a package should not require users to
navigate to a particular directory to do their work. ``Where I have to
be'' is just another hard-coded path.

In order to save typing, it is often convenient to allow users to
specify an input or output \emph{directory}, and then require that there
be files with particular names in that directory. This practice, which
is sometimes called ``convention over configuration'', is used by many
software frameworks, such as WordPress and Ruby on Rails, and often
strikes a good balance between adaptability and consistency.

\section{Reuse software (without tears).}

In the spirit of code reuse and interoperability, developers often want
to use software written by others. The tool could be standard for the
task at hand or do exactly what is necessary. With a few lines, a call
is made out to the other program, the results are incorporated into the
primary script. Using popular projects reduces the amount of code that
needs to be maintained and adds the strength of vetted software to the
final program.

Unfortunately, the interface between two software packages can be a
source of considerable frustration. Support requests descend into
debugging errors produced by the other project.

The way that the second program is invoked can throw up errors. For
example, the program may not be in the user's path, or it may be an
older or newer version and produce results differen than expected.
Windows users will be frustrated if you invoke bash or sh explictly or
use shell-specific conventions like `*' expansion. It is unlikely that
all of your users will all be on the operating system and version as
you. Even Linux-standard functions available vary slightly between
installs. For example: GNU \texttt{sort} is available on almost every
*nix distribution, but sorts differently depending on locale.

We fully support the reuse of software between projects and have some
suggestions to reduce the aforementioned pains. First, ensure the
appropriate software and version is available. Either allow the user to
configure the exact path to the package, distribute the program with the
dependent software, or download it during installation using a
dependency management system. Regardless, check whether the program is
executable and what version is running. Be sure to remind users in the
documentation what the compatible versions are.

Second, to ensure support on as many different operating systems as
possible, use native functions for starting other processes, such as
Java's Runtime.exec call, Python's subprocess module, and Perl's system
command, and be sure to capture and report the standard error output of
the subprocess to facilitate debugging.

Finally, make sure that you really need the call-out. If you are
executing GNU sort instead of figuring out how to sort lists in Python,
it may not be worth the tears of reuse.

\section{Do not rely on the pre-installation of non-standard packages or libraries unless clearly stated in the documentaton.}

Every package someone has to install before being able to use yours is a
possible (some would say ``likely'') source of frustration for some
potential user. On the other hand, research software developers should
re-use existing software wherever possible. To strike a balance between
these two, developers should document \emph{all} of the packages that
theirs depends on, preferably in a machine-readable form. For example,
it is common for Python projects to include a file called
\texttt{requirements.txt} that lists the names of required libraries,
along with version ranges:

\begin{verbatim}
requests>=2.0
pygithub>=1.26,<=1.27
python-social-auth>=0.2.19,<0.3
\end{verbatim}

This file can be read by a package manager, which can check that the
required software is available, and install it if it is not. Similar
mechanisms exist for Perl, R, and other languages.

A common way to break this rule is to depend on scripts and tools that
are installed on the computers the original developer is using, but
which aren't otherwise packaged and available. In many cases, the author
of a package may not realize that some tool was built locally, and
doesn't exist elsewhere. At present, the only sure way to discover such
unknown dependencies is to install on a system administered by someone
else and see what breaks. In future, as use of lightweight
virtualization containers like Docker becomes more widespread, it may
become common to test installation on a virtual machine.

\section{Produce identical results when given identical inputs.}

Given a set of parameters and a dataset, the package should produce the
same results every time it is run.

Many bioinformatics applications rely on randomized algorithms to
improve performance or runtimes. As a consequence, results can change
between runs, even when provided with the same data and parameters. By
its nature, this randomness renders strict reproducibility impossible.
Debugging is more difficult. If even the small test set (\#9) produces
different results, new users may be able to tell whether the software is
working properly, eroding confidence in the application. When comparing
results between versions or after changing parameters, even small
differences can confuse or muddy the comparison. And especially when
producing results for publications, grants or diagnoses, any analysis
should be absolutely reproducible.

Given the size of biological data, it is unreasonable to suggest that
random algorithms be removed. However, most programs use a pseudo-random
number generator, which uses a starting seed and an equation to
approximate random numbers. Setting the seed to a consistent value
removes randomness between runs. Allow the user to optionally provide
the seed as an input parameter, thus rendering the program deterministic
for those cases where it matters. If the seed is set internally (e.g.,
using clock time), echo it to the output for re-use later.

\section{Include a small test set that can be run to ensure the software is actually working.}

Every package should come with a small test script for users to run
after installation. Its purpose is \emph{not} to check that the software
is working correctly (although that is extremely helpful), but rather to
ensure that it will work at all. This test script can also serve as a
working example of how to run the software, which is valuable in case
its documentation has fallen out of sync with recent changes to the code
itself.

In order to be useful, this test script must be easy to find and run. A
single file in the project's root directory named \texttt{runtests.sh}
or something equally obvious is a much better solution than documenting
test cases and requiring people to copy and paste them into the shell.

Equally, the test script's output must be easy to interpret. Screens
full of correlation coefficients do not qualify: instead, the script's
output should be something like one line per test, with the test's name
and its pass/fail status, followed by a single summary line saying how
many tests were run and how many passed or failed. If many or all tests
fail because of missing dependencies, that fact should be displayed
once, clearly, rather than once per test, so that users have a clear
idea of what they need to fix and how much work it's likely to take.

\section{Give the software a meaningful version number.}

Most software has a version number composed of a decimal number that
increments as new versions are released. There are many different ways
to construct and interpret the version, but most importantly for us, a
particular software version run with the same parameters should give
identical results no matter when it's run. Results include correct
output as well as any errors, whether they arise from incorrect input or
are bugs. Every time you release your software, i.e.~distribute it to
someone other than yourself and/or the development team, you should
increment your version number.

\withurl{Semantic versioning}{http://semver.org/} is one of the most common
types of versioning for open-source software. Version numbers take the
form of \emph{MAJOR.MINOR{[}.PATCH{]}}, e.g.~0.2.6-RC1. The major and
minor numbers versions are almost always provided. Changes in the major
version number herald significant changes in the software that are not
backwards compatible, such as changing or removing features or altering
the primary functions of the software. Increasing the minor version
represents incremental improvements in the software like adding new
features. Following the minor version number can be an arbitrary number
of project-specific versions, including patches, builds and qualifiers.
Common qualifiers include \texttt{-SNAPSHOT}, for applications that are
not yet stable or released, and \texttt{-RC} for release candidate prior
to official release.

The version of your software should be easily available, both when
supplying \texttt{-\/-version} or \texttt{-v} on the command line as
well as in the results. The software version should be printed to the
same location as all of the other parameters (see Rule 4).

Old versions of your software should be available to ensure that results
are reproducible into the far future. A number of mechanisms exist for
controlled release that range from as simple as adding an appropriate
commit message or tag to version control, to official releases alongside
code on Sourceforge, Bitbutcket or Github, to depositing into a
repository like apt, yum, homebrew, CPAN, etc. Choose the method that
best suits the number and expertise of users you anticipate.

\section*{Conclusion}

\fixme{GW}{How to tell if the software is working:}

\begin{itemize}
\item
  Test in a `vanilla' environment, such as another user's computer or a
  dummy account with none of the settings of the original developer.
\item
  Test with different sizes of data: ridiculously small, small, medium,
  and large. The software should run on all sizes given some parameter
  tweaking, or fail with a sensible error message if the data is too
  big.
\item
  Compare results from multiple iterations that have the same parameters
  and inputs.(See rule \#8.)
\item
  Consider a container - but most of these rules still apply
\end{itemize}

\bibliography{robust-software}

\end{document}
